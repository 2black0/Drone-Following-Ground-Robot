\documentclass[%
 aip,
% jmp,
% bmf,
% sd,
% rsi,
cp,  % Conference Proceedings
 amsmath,amssymb,%nobibnotes,
% preprint,%
 reprint,%
%author-year,%
%author-numerical,%
]{revtex4-2}

\usepackage{graphicx}% Include figure files
\usepackage{dcolumn}% Align table columns on decimal point

\usepackage{bm}% bold math
%\usepackage[mathlines]{lineno}% Enable numbering of text and display math
%\linenumbers\relax % Commence numbering lines

\usepackage[utf8]{inputenc}
\usepackage[T1]{fontenc}
%% Loads a Times-like font. You can also load
%% {newtxtext,newtxtmath}, but not {times}, 
%% {txfonts} nor {mathtpm} as these packages
%% are obsolete and have been known to cause problems.
\usepackage{mathptmx} 

\begin{document}

\title{Marker-Based Autonomous Quadrotor Tracking for Ground Mobile Robots}% Force line breaks with \\

\author{Ardy Seto Priambodo} % Write as First name Surname
 \email[Corresponding author: ]{ardyseto@uny.ac.id}
\author{Oktaf Agni Dhewa}%
 \email{oktafagnidhewa@uny.ac.id}
\author{Aris Nasuha}%
 \email{arisnasuha@uny.ac.id}
\affiliation{
  1Electrical and Electronics Engineering Dept., Vocational Faculty, Universitas Negeri Yogyakarta, Indonesia % Force line breaks with \\ if necessary
}

\author{Fatchul Arifin}
 \email{fatchul@uny.ac.id}
\author{Anggun Winursito}
 \email{anggunwinursito@uny.ac.id }
\affiliation{%
2Electronics and Informatics Engineering Education Dept., Engineering Faculty, Universitas Negeri Yogyakarta, Indonesia % Force line breaks with \\ if necessary
}%

\date{\today} % It is always \today, today, but any date may be explicitly specified
              % Not printed for conference proceedings

\begin{abstract}
An article usually includes an abstract, a concise summary of the work
covered at length in the main body of the article. It is used for
secondary publications and for information retrieval purposes.
\end{abstract}

\maketitle

\section{Introduction}

In recent years, the utilization of unmanned aerial vehicles, commonly known as drones or quadcopters, has surged across a wide spectrum of applications, ranging from aerial photography and surveillance to search and rescue operations. Drones have shown their potential to revolutionize various industries because of their ability to operate autonomously, collect real-time data, and navigate through challenging environments. This study explores an innovative approach to enhance the capabilities of ground robots through the implementation of an autonomous drone-following system using ArUco markers.

Quadcopters, or drones, have gained substantial prominence owing to their agility and versatility in navigating complex terrain. These aerial platforms are equipped with multirotor configurations that provide impressive stability and maneuverability. Their potential lies not only in aerial reconnaissance but also in their ability to collaborate with ground robots, creating dynamic and adaptive robotic teams. This paper presents a novel application of quadcopters in the context of ground robot interaction, enabling the seamless following of ground-based robots through the integration of ArUco markers.

Although previous research has explored aspects of both autonomous drone operation and ground robot navigation, there is a relatively limited body of work focusing on the intersection of these two domains. Existing studies have often employed GPS or vision-based techniques for tracking ground robots; however, these methods may be less reliable in GPS-denied environments or when visual obstructions are present. This study addresses the existing gap by proposing an autonomous drone-following system that leverages ArUco markers, providing a cost-effective and robust solution for enhancing the coordination between aerial and ground-based robotic platforms. By utilizing ArUco markers as visual cues, the proposed system aims to overcome the limitations of traditional tracking methods, thereby offering greater accuracy and adaptability in challenging real-world scenarios. This study strives to advance the field of autonomous robotics by introducing an innovative approach to enhancing collaborative robotics through the integration of drone technology and marker-based tracking mechanisms.

\section{Methodology}

\subsection{Model of Quadrotor}
A quadcopter is a popular unmanned aerial vehicle (UAV) characterized by four rotors arranged in a cross configuration. Understanding this basic model is fundamental for designing control algorithms and tracking systems. The motion of the quadcopter can be divided into six degrees of freedom: three translational and three rotational. Translational motion refers to the movement of the vehicle along the X-, Y-, and Z-axes in 3D space, whereas rotational motion represents its orientation about these axes.

In terms of the quadcopter's basic kinematics, its translational motion is governed by Newton's second law, where the thrust generated by the four rotors must balance the gravitational force and account for any desired acceleration. On the other hand, rotational motion follows the principles of angular momentum conservation, where the torque generated by the rotors creates angular acceleration. This simple yet effective model allows us to accurately analyze and control the quadcopter’s movements.

The dynamics of a quadcopter involve a complex interplay of the forces and torques acting on the vehicle. These dynamics are typically described by a set of differential equations that consider the mass, geometry, and aerodynamic properties of the rotors of the vehicle. In a simplified model, the dynamics of a quadcopter can be represented by a system of equations that govern its translational and rotational motion.

For translational dynamics, we considered the forces generated by the four rotors and their effect on the vehicle's linear acceleration. These forces are determined by the throttle inputs to the individual rotors and are subject to nonlinearities, such as thrust losses owing to air resistance.

Rotational dynamics involve the torques generated by the rotors and their effect on the quadcopter's angular acceleration. The moments of inertia and geometry of the quadcopter play crucial roles in these equations. Additionally, aerodynamic effects, such as rotor-induced drag and gyroscopic precession, must be considered for an accurate model. These equations provide the foundation for developing control strategies to stabilize and maneuver a quadcopter in a controlled manner. The dynamic equation of the quadrotor is shown as follows:

\begin{eqnarray}
\ddot{\phi} = \dot{\theta}\dot{\psi}\left ( \frac{I_y-I_z}{I_x} \right ) - \frac{J_r}{I_x}\dot{\theta}\Omega + \frac{l}{I_x}U_2 \\
\ddot{\theta} = \dot{\phi}\dot{\psi}\left ( \frac{I_z-I_x}{I_y} \right ) - \frac{J_r}{I_y}\dot{\phi}\Omega + \frac{l}{I_y}U_3 \\
\ddot{\psi} = \dot{\phi}\dot{\theta}\left ( \frac{I_x-I_y}{I_z} \right ) + \frac{l}{I_z}U_4 \\
\ddot{x} = (cos \phi sin \theta cos \psi + sin \phi sin \psi) \frac{U_1}{m} \\
\ddot{y} = (cos \phi sin \theta cos \psi - sin \phi sin \psi) \frac{U_1}{m} \\
\ddot{z} = -g (cos \phi cos \theta) \frac{U_1}{m}
\end{eqnarray}

\subsection{ArUco Marker}
In the realm of autonomous systems, marker-based tracking has emerged as a robust and efficient method for enhancing the localization and navigation capabilities of aerial vehicles like quadrotors in conjunction with ground mobile robots. One prominent marker system that has gained significant attention in recent years is the ArUco marker. ArUco markers are a type of fiducial marker characterized by their simplicity, ease of detection, and robustness. In this section, we delve into the explanation of ArUco markers and the algorithms employed for their detection, with a focus on the utilization of OpenCV for this purpose.

Aruco markers, short for Augmented Reality University of Cordoba markers, are a type of visual marker designed for computer vision applications. They consist of a grid of black squares on a white background, with a unique identifier encoded within. These markers are typically printed and affixed to objects or surfaces that need to be tracked. What makes ArUco markers particularly advantageous is their high detection accuracy, even in challenging lighting conditions and varying viewing angles. The unique identifier within each marker allows for easy differentiation, making them an ideal choice for robotics applications where precise localization and tracking are essential.

The process of detecting ArUco markers often involves computer vision libraries such as OpenCV, which provides a rich set of tools for image processing and marker recognition. OpenCV offers a specialized ArUco module that simplifies marker detection and pose estimation. The algorithm typically employed for ArUco marker detection comprises several key steps, including image preprocessing, contour detection, corner detection, and marker identification. OpenCV's ArUco module incorporates methods to estimate the marker's pose in three-dimensional space, enabling accurate localization of the robot or quadrotor relative to the markers in the environment.


\section{Proposed Algorithm}


\section{Simulation Environment Setup}

\subsection{Ground Robot}

\subsection{Quadrotor Robot}

\section{Experimental Result}

\subsection{Marker Detection}

\subsection{Tracking Accuracy and Reliability}

\section{Conclusion}

\iffalse
This sample document demonstrates proper use of REV\TeX~4.1 (and
\LaTeXe) in manuscripts prepared for submission to AIP
conference proceedings. Further information can be found in the documentation included in the distribution or available at
\url{http://authors.aip.org} and in the documentation for
REV\TeX~4.1 itself.

When commands are referred to in this example file, they are always
shown with their required arguments, using normal \TeX{} format. In
this format, \verb+#1+, \verb+#2+, etc. stand for required
author-supplied arguments to commands. For example, in
\verb+\section{#1}+ the \verb+#1+ stands for the title text of the
author's section heading, and in \verb+\title{#1}+ the \verb+#1+
stands for the title text of the paper.

Line breaks in section headings at all levels can be introduced using
\textbackslash\textbackslash. A blank input line tells \TeX\ that the
paragraph has ended.

\subsection{\label{sec:level2}Second-level heading: Formatting}

This file may be formatted in both the \texttt{preprint} (the default) and
\texttt{reprint} styles; the latter format may be used to
mimic final journal output. Either format may be used for submission
purposes. Hence, it is essential that authors check that their manuscripts format acceptably
under \texttt{preprint}. Manuscripts submitted to AIP that do not
format correctly under the \texttt{preprint} option may be delayed in
both the editorial and production processes.

\subsubsection{\label{sec:level3}Third-level heading: Citations and Footnotes}

Citations in text refer to entries in the Bibliography;
they use the commands \verb+\cite{#1}+ or \verb+\onlinecite{#1}+.
Because REV\TeX\ uses the \verb+natbib+ package of Patrick Daly,
its entire repertoire of commands are available in your document;
see the \verb+natbib+ documentation for further details.
The argument of \verb+\cite+ is a comma-separated list of \emph{keys};
a key may consist of letters and numerals.

By default, citations are numerical; \cite{feyn54} author-year citations are an option.
To give a textual citation, use \verb+\onlinecite{#1}+: (Refs.~\onlinecite{witten2001,epr,Bire82}).
REV\TeX\ ``collapses'' lists of consecutive numerical citations when appropriate.
REV\TeX\ provides the ability to properly punctuate textual citations in author-year style;
this facility works correctly with numerical citations only with \texttt{natbib}'s compress option turned off.
To illustrate, we cite several together \cite{feyn54,witten2001,epr,Berman1983},
and once again (Refs.~\onlinecite{epr,feyn54,Bire82,Berman1983}).
Note that, when numerical citations are used, the references were sorted into the same order they appear in the bibliography.

A reference within the bibliography is specified with a \verb+\bibitem{#1}+ command,
where the argument is the citation key mentioned above.
\verb+\bibitem{#1}+ commands may be crafted by hand or, preferably,
generated by using Bib\TeX.
The AIP styles for REV\TeX~4 include Bib\TeX\ style files
\verb+aipnum.bst+ and \verb+aipauth.bst+, appropriate for
numbered and author-year bibliographies,
respectively.
REV\TeX~4 will automatically choose the style appropriate for
the document's selected class options: the default is numerical, and
you obtain the author-year style by specifying a class option of \verb+author-year+.

This sample file demonstrates a simple use of the Bib\TeX\ tool
via a \verb+\bibliography+ command referencing the \verb+aipsamp.bib+ file.
Running Bib\TeX\ (in this case \texttt{bibtex
aipsamp}) after the first pass of \LaTeX\ produces the file
\verb+aipsamp.bbl+ which contains the automatically formatted
\verb+\bibitem+ commands (including extra markup information via
\verb+\bibinfo+ commands). If not using Bib\TeX, the
\verb+thebibiliography+ environment should be used instead.

\paragraph{Fourth-level heading is run in.}%
Footnotes are produced using the \verb+\footnote{#1}+ command.
Numerical style citations put footnotes into the
bibliography\footnote{Automatically placing footnotes into the bibliography requires using BibTeX to compile the bibliography.}.
Author-year and numerical author-year citation styles (each for its own reason) cannot use this method.
Note: due to the method used to place footnotes in the bibliography, \emph{you
must re-run BibTeX every time you change any of your document's
footnotes}.

\section{Math and Equations}
Inline math may be typeset using the \verb+$+ delimiters. Bold math
symbols may be achieved using the \verb+bm+ package and the
\verb+\bm{#1}+ command it supplies. For instance, a bold $\alpha$ can
be typeset as \verb+$\bm{\alpha}$+ giving $\bm{\alpha}$. Fraktur and
Blackboard (or open face or double struck) characters should be
typeset using the \verb+\mathfrak{#1}+ and \verb+\mathbb{#1}+ commands
respectively. Both are supplied by the \texttt{amssymb} package. For
example, \verb+$\mathbb{R}$+ gives $\mathbb{R}$ and
\verb+$\mathfrak{G}$+ gives $\mathfrak{G}$

In \LaTeX\ there are many different ways to display equations, and a
few preferred ways are noted below. Displayed math will center by
default. Use the class option \verb+fleqn+ to flush equations left.

Below we have numbered single-line equations, the most common kind:
\begin{eqnarray}
\chi_+(p)\alt{\bf [}2|{\bf p}|(|{\bf p}|+p_z){\bf ]}^{-1/2}
\left(
\begin{array}{c}
|{\bf p}|+p_z\\
px+ip_y
\end{array}\right)\;,
\\
\left\{%
 \openone234567890abc123\alpha\beta\gamma\delta1234556\alpha\beta
 \frac{1\sum^{a}_{b}}{A^2}%
\right\}%
\label{eq:one}.
\end{eqnarray}
Note the open one in Eq.~(\ref{eq:one}).

Not all numbered equations will fit within the text width this
way. The equation number will move down automatically if it cannot fit
on the same line with a one-line equation:
\begin{equation}
\left\{
 ab12345678abc123456abcdef\alpha\beta\gamma\delta1234556\alpha\beta
 \frac{1\sum^{a}_{b}}{A^2}%
\right\}\left(1234567890abcdefghijklmnopqrstuvwxyz123456789\right).
\end{equation}

When the \verb+\label{#1}+ command is used [cf. input for
Eq.~(\ref{eq:one})], the equation can be referred to in text without
knowing the equation number that \TeX\ will assign to it. Just
use \verb+\ref{#1}+, where \verb+#1+ is the same name that used in
the \verb+\label{#1}+ command.

Unnumbered single-line equations can be typeset
using the \verb+\[+, \verb+\]+ format:
\[g^+g^+ \rightarrow g^+g^+g^+g^+ \dots ~,~~q^+q^+\rightarrow
q^+g^+g^+ \dots ~. \]

\subsection{Multiline equations}

Multiline equations are obtained by using the \verb+eqnarray+
environment.  Use the \verb+\nonumber+ command at the end of each line
to avoid assigning a number:
\begin{eqnarray}
{\cal M}=&&ig_Z^2(4E_1E_2)^{1/2}(l_i^2)^{-1}
\delta_{\sigma_1,-\sigma_2}
(g_{\sigma_2}^e)^2\chi_{-\sigma_2}(p_2)\nonumber\\
&&\times
[\epsilon_jl_i\epsilon_i]_{\sigma_1}\chi_{\sigma_1}(p_1),
\end{eqnarray}
\begin{eqnarray}
\sum \vert M^{\text{viol}}_g \vert ^2&=&g^{2n-4}_S(Q^2)~N^{n-2}
        (N^2-1)\nonumber \\
 & &\times \left( \sum_{i<j}\right)
  \sum_{\text{perm}}
 \frac{1}{S_{12}}
 \frac{1}{S_{12}}
 \sum_\tau c^f_\tau~.
\end{eqnarray}
\textbf{Note:} Do not use \verb+\label{#1}+ on a line of a multiline
equation if \verb+\nonumber+ is also used on that line. Incorrect
cross-referencing will result. Notice the use \verb+\text{#1}+ for
using a Roman font within a math environment.

To set a multiline equation without \emph{any} equation
numbers, use the \verb+\begin{eqnarray*}+,
\verb+\end{eqnarray*}+ format:
\begin{eqnarray*}
\sum \vert M^{\text{viol}}_g \vert ^2&=&g^{2n-4}_S(Q^2)~N^{n-2}
        (N^2-1)\\
 & &\times \left( \sum_{i<j}\right)
 \left(
  \sum_{\text{perm}}\frac{1}{S_{12}S_{23}S_{n1}}
 \right)
 \frac{1}{S_{12}}~.
\end{eqnarray*}
To obtain numbers not normally produced by the automatic numbering,
use the \verb+\tag{#1}+ command, where \verb+#1+ is the desired
equation number. For example, to get an equation number of
(\ref{eq:mynum}),
\begin{equation}
g^+g^+ \rightarrow g^+g^+g^+g^+ \dots ~,~~q^+q^+\rightarrow
q^+g^+g^+ \dots ~. \tag{5.1$'$}\label{eq:mynum}
\end{equation}

A few notes on \verb=\tag{#1}=. \verb+\tag{#1}+ requires
\texttt{amsmath}. The \verb+\tag{#1}+ must come before the
\verb+\label{#1}+, if any. The numbering set with \verb+\tag{#1}+ is
\textit{transparent} to the automatic numbering in REV\TeX{};
therefore, the number must be known ahead of time, and it must be
manually adjusted if other equations are added. \verb+\tag{#1}+ works
with both single-line and multiline equations. \verb+\tag{#1}+ should
only be used in exceptional case - do not use it to number all
equations in a paper.

Enclosing single-line and multiline equations in
\verb+\begin{subequations}+ and \verb+\end{subequations}+ will produce
a set of equations that are ``numbered'' with letters, as shown in
Eqs.~(\ref{subeq:1}) and (\ref{subeq:2}) below:
\begin{subequations}
\label{eq:whole}
\begin{equation}
\left\{
 abc123456abcdef\alpha\beta\gamma\delta1234556\alpha\beta
 \frac{1\sum^{a}_{b}}{A^2}
\right\},\label{subeq:1}
\end{equation}
\begin{eqnarray}
{\cal M}=&&ig_Z^2(4E_1E_2)^{1/2}(l_i^2)^{-1}
(g_{\sigma_2}^e)^2\chi_{-\sigma_2}(p_2)\nonumber\\
&&\times
[\epsilon_i]_{\sigma_1}\chi_{\sigma_1}(p_1).\label{subeq:2}
\end{eqnarray}
\end{subequations}
Putting a \verb+\label{#1}+ command right after the
\verb+\begin{subequations}+, allows one to
reference all the equations in a subequations environment. For
example, the equations in the preceding subequations environment were
Eqs.~(\ref{eq:whole}).

\section{Cross-referencing}
REV\TeX{} will automatically number sections, equations, figure
captions, and tables. In order to reference them in text, use the
\verb+\label{#1}+ and \verb+\ref{#1}+ commands. To reference a
particular page, use the \verb+\pageref{#1}+ command.

The \verb+\label{#1}+ should appear in a section heading, within an
equation, or in a table or figure caption. The \verb+\ref{#1}+ command
is used in the text where the citation is to be displayed.  Some
examples: Section~\ref{sec:level1} on page~\pageref{sec:level1},
Table~\ref{tab:table1},%
\begin{table}
\caption{\label{tab:table1}This table illustrates left-aligned, centered,
and right-aligned columns. Note that REV\TeX~4 adjusts the
intercolumn spacing so that the table fills the entire width of the text.
Table captions are numbered automatically.  }
\begin{ruledtabular}
\begin{tabular}{lcr}
Left\footnote{Note a.}&Centered\footnote{Note b.}&Right\\
\hline
1 & 2 & 3\\
10 & 20 & 30\\
100 & 200 & 300\\
\end{tabular}
\end{ruledtabular}
\end{table}
and Fig.~\ref{fig:epsart}.

\section{Figures and Tables}
Figures and tables are typically ``floats''; \LaTeX\ determines their
final position via placement rules.
\LaTeX\ isn't always successful in automatically placing floats where you wish them.

Figures are marked up with the \texttt{figure} environment, the content of which
imports the image (\verb+\includegraphics+) followed by the figure caption (\verb+\caption+).
The argument of the latter command should itself contain a \verb+\label+ command if you
wish to refer to your figure with \verb+\ref+.

Import your image using either the \texttt{graphics} or
\texttt{graphicx} packages. Both of these packages define the
\verb+\includegraphics{#1}+ command, but they differ in the optional
arguments for specifying the orientation, scaling, and translation of the figure.
Fig.~\ref{fig:epsart}%
\begin{figure}
\includegraphics{fig_1}% Here is how to import EPS art
\caption{\label{fig:epsart} A figure caption. The figure captions are
automatically numbered.}
\end{figure}
is an example of this.

The analog of the \texttt{figure} environment is \texttt{table}, which uses
the same \verb+\caption+ command.
However, you should type your caption command first within the \texttt{table},
instead of last as you did for \texttt{figure}.

The heart of any table is the \texttt{tabular} environment,
which represents the table content as a (vertical) sequence of table rows,
each containing a (horizontal) sequence of table cells.
Cells are separated by the \verb+&+ character;
the row terminates with \verb+\\+.
The required argument for the \texttt{tabular} environment
specifies how data are displayed in each of the columns.
For instance, a column
may be centered (\verb+c+), left-justified (\verb+l+), right-justified (\verb+r+),
or aligned on a decimal point (\verb+d+).
(Table~\ref{tab:table4}%
\begin{table}
\caption{\label{tab:table4}Numbers in columns Three--Five have been
aligned by using the ``d'' column specifier (requires the
\texttt{dcolumn} package).
Non-numeric entries (those entries without
a ``.'') in a ``d'' column are aligned on the decimal point.
Use the
``D'' specifier for more complex layouts. }
\begin{ruledtabular}
\begin{tabular}{ccddd}
One&Two&\mbox{Three}&\mbox{Four}&\mbox{Five}\\
\hline
one&two&\mbox{three}&\mbox{four}&\mbox{five}\\
He&2& 2.77234 & 45672. & 0.69 \\
C\footnote{Some tables require footnotes.}
  &C\footnote{Some tables need more than one footnote.}
  & 12537.64 & 37.66345 & 86.37 \\
\end{tabular}
\end{ruledtabular}
\end{table}
illustrates the use of decimal column alignment.)

Extra column-spacing may be be specified as well, although
REV\TeX~4 sets this spacing so that the columns fill the width of the
table.
Horizontal rules are typeset using the \verb+\hline+
command.
The doubled (or Scotch) rules that appear at the top and
bottom of a table can be achieved by enclosing the \texttt{tabular}
environment within a \texttt{ruledtabular} environment.
Rows whose columns span multiple columns can be typeset using \LaTeX's
\verb+\multicolumn{#1}{#2}{#3}+ command
(for example, see the first row of Table~\ref{tab:table3}).%
\begin{table}
\caption{\label{tab:table3}This table demonstrates the use of
\textbackslash\texttt{multicolumn} in rows with entries that span
more than one column.}
\begin{ruledtabular}
\begin{tabular}{ccccc}
 &\multicolumn{2}{c}{$D_{4h}^1$}&\multicolumn{2}{c}{$D_{4h}^5$}\\
 Ion&1st alternative&2nd alternative&lst alternative
&2nd alternative\\ \hline
 K&$(2e)+(2f)$&$(4i)$ &$(2c)+(2d)$&$(4f)$ \\
 Mn&$(2g)$\footnote{The $z$ parameter of these positions is $z\sim\frac{1}{4}$.}
 &$(a)+(b)+(c)+(d)$&$(4e)$&$(2a)+(2b)$\\
 Cl&$(a)+(b)+(c)+(d)$&$(2g)$\footnote{This is a footnote in a table. It is supposed to set on the full width of the page, just as the caption does. }
 &$(4e)^{\text{a}}$\\
 He&$(8r)^{\text{a}}$&$(4j)^{\text{a}}$&$(4g)^{\text{a}}$\\
 Ag& &$(4k)^{\text{a}}$& &$(4h)^{\text{a}}$\\
\end{tabular}
\end{ruledtabular}
\end{table}

The tables in this document illustrate various effects.
Lengthy tables may need to break across pages.
A simple way to allow this is to specify
the \verb+[H]+ float placement on the \texttt{table} environment.
Alternatively, using the standard \LaTeXe\ package \texttt{longtable}
gives more control over how tables break and allows headers and footers
to be specified for each page of the table.
An example of the use of \texttt{longtable} can be found
in the file \texttt{summary.tex} that is included with the REV\TeX~4
distribution.

There are two methods for setting footnotes within a table (these
footnotes will be displayed directly below the table rather than at
the bottom of the page or in the bibliography).
The easiest
and preferred method is just to use the \verb+\footnote{#1}+
command. This will automatically enumerate the footnotes with
lowercase roman letters.
However, it is sometimes necessary to have
multiple entries in the table share the same footnote.
In this case,
create the footnotes using
\verb+\footnotemark[#1]+ and \verb+\footnotetext[#1]{#2}+.
\texttt{\#1} is a numeric value.
Each time the same value for \texttt{\#1} is used,
the same mark is produced in the table.
The \verb+\footnotetext[#1]{#2}+ commands are placed after the \texttt{tabular}
environment.
Examine the \LaTeX\ source and output for Tables~\ref{tab:table1} and
\ref{tab:table2}%
\begin{table}
\caption{\label{tab:table2}A table with more columns still fits
properly in a column. Note that several entries share the same
footnote. Inspect the \LaTeX\ input for this table to see
exactly how it is done.}
\begin{ruledtabular}
\begin{tabular}{cccccccc}
 &$r_c$ (\AA)&$r_0$ (\AA)&$\kappa r_0$&
 &$r_c$ (\AA) &$r_0$ (\AA)&$\kappa r_0$\\
\hline
Cu& 0.800 & 14.10 & 2.550 &Sn\footnotemark[1]
& 0.680 & 1.870 & 3.700 \\
Ag& 0.990 & 15.90 & 2.710 &Pb\footnotemark[2]
& 0.450 & 1.930 & 3.760 \\
Au& 1.150 & 15.90 & 2.710 &Ca\footnotemark[3]
& 0.750 & 2.170 & 3.560 \\
Mg& 0.490 & 17.60 & 3.200 &Sr\footnotemark[4]
& 0.900 & 2.370 & 3.720 \\
Zn& 0.300 & 15.20 & 2.970 &Li\footnotemark[2]
& 0.380 & 1.730 & 2.830 \\
Cd& 0.530 & 17.10 & 3.160 &Na\footnotemark[5]
& 0.760 & 2.110 & 3.120 \\
Hg& 0.550 & 17.80 & 3.220 &K\footnotemark[5]
&  1.120 & 2.620 & 3.480 \\
Al& 0.230 & 15.80 & 3.240 &Rb\footnotemark[3]
& 1.330 & 2.800 & 3.590 \\
Ga& 0.310 & 16.70 & 3.330 &Cs\footnotemark[4]
& 1.420 & 3.030 & 3.740 \\
In& 0.460 & 18.40 & 3.500 &Ba\footnotemark[5]
& 0.960 & 2.460 & 3.780 \\
Tl& 0.480 & 18.90 & 3.550 & & & & \\
\end{tabular}
\end{ruledtabular}
\footnotetext[1]{Here's the first, from Ref.~\onlinecite{feyn54}.}
\footnotetext[2]{Here's the second.}
\footnotetext[3]{Here's the third.}
\footnotetext[4]{Here's the fourth.}
\footnotetext[5]{And etc.}
\end{table}
for an illustration.

Sometimes it can be convenient to place multiple narrow figures or tables side-by-side to conserve space and meet any page length requirements for your conference. This can be done using \texttt{minipage} environments within the \texttt{table} or \texttt{figure} environment. Check the \LaTeX\ source and output for Tables~\ref{tab:multiples1}, \ref{tab:multiples2}, and \ref{tab:multiples3}
\begin{table}[tbp]
\begin{minipage}[t]{0.3\textwidth}
    \caption{First narrow table.}
    \label{tab:multiples1}
    \begin{ruledtabular}
    \begin{tabular}{ll}
        Element Symbol & Element Name \\
        \hline
        H & Hydrogen \\
    \end{tabular}
    \end{ruledtabular}
\end{minipage}\hfill%
\begin{minipage}[t]{0.3\textwidth}
    \caption{Second narrow table, set alongside.}
    \label{tab:multiples2}
    \begin{ruledtabular}
    \begin{tabular}{lr}
        Trial & Time (s) \\
        \hline
        1 & 2.42 \\
        2 & 2.46 \\
        3 & 2.41 \\
    \end{tabular}
    \end{ruledtabular}
\end{minipage}\hfill%
\begin{minipage}[t]{0.3\textwidth}
    \caption{Third narrow table, set alongside once again.}
    \label{tab:multiples3}
    \begin{ruledtabular}
    \begin{tabular}{ll}
        Case & Result \\
        \hline
        A & Pass \\
        B & Fail \\
        C & Pass \\
    \end{tabular}
    \end{ruledtabular}
\end{minipage}
\end{table}
for an example of how to do this. The vertical alignment of the \texttt{minipage}s can be adjusted by changing the optional argument to the environment.

All AIP journals require that the initial citation of
figures or tables be in numerical order.
\LaTeX's automatic numbering of floats is your friend here:
just put each \texttt{figure} environment immediately following
its first reference (\verb+\ref+), as we have done in this example file.

\section{Conclusion}

In this section we welcome you to include a summary of the end results of your research.

\begin{acknowledgments}
We wish to acknowledge the support of the author community in using
REV\TeX{}, offering suggestions and encouragement, testing new versions,
\dots.
\end{acknowledgments}

\fi

\nocite{*}
\bibliography{aipsamp}% Produces the bibliography via BibTeX.

\end{document}
%
% ****** End of file aipsamp.tex ******
